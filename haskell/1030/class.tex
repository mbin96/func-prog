\documentclass{article}
\usepackage[utf8]{inputenc}
\usepackage{kotex}
% for the fancy \koTeX logo
\usepackage{kotex-logo}

\usepackage{minted}
%-shell-escape
\usepackage{natbib}
\usepackage{graphicx}

\begin{document}

\title{hello haskell}
\author{건국대학교 조현종}
\date{October 30 2019}

\maketitle

\section{Install Haskell on Ubuntu}
하스켈을 사용하기 위해서는 터미널창의 다음과 같은 명령어를 입력한다.
\begin{minted}[breaklines]{bash}
    $ sudo apt-get install haskell-platform
\end{minted}


\section{Using GHCi}
설치가 완료되면, 터미널에 ghci를 입력해보자. 

\begin{minted}[breaklines]{bash}
    $ ghci
\end{minted}
다음과 같은 화면이 나올것이다.\\
GHCi, version 8.0.2: http://www.haskell.org/ghc/  :? for help \\
Prelude> \\
GHCi 라는 Haskell 컴파일러 인터프리터를 실행했다. Prelude> 는 가장 기본적임 함수들이 정의된 모듈이다. 앞으로 우리는 하스켈로 모듈을 직접 작성하여 함수를 사용해 볼 것이다. \\
그전에 여러가지 GHCi 명령어에 익숙해져보자.

\begin{minted}[breaklines]{haskell}
Prelude> :t 3
3 :: Num t => t
Prelude> :t 3+5
3+5 :: Num a => a
Prelude> :t 3.4
3.4 :: Fractional t => t
Prelude> :t pi
pi :: Floating a => a
Prelude> :t [1,2,3]
[1,2,3] :: Num t => [t]
Prelude> :t \x -> x
\x -> x :: t -> t
Prelude> \x -> x
\end{minted}
:t 명령어는 뒤에 오는 변수나 함수의 타입을 알려준다. 3 은 Num타입으로 추론되며, 자기 자신을 리턴한다. 3.4는 분수로 표현가능하므로 Fractional타입으로 추론된다. pi($\pi$)는 부동소수점이므로 Floating
으로 추론된다.\\
:edit 명령어는 현재 실행중인 모듈 파일을 실행하여 편집을 할수있게 한다. \\
:set editor <editor name> 명령어는 :edit 명령을 실행할때 실행할 에디터를 선택한다. vim, gedit, nano, code(vscode) 등의 자신이 사용하기 편한 cui 혹은 gui 에디터를 사용하면 된다. \\
:cd <directory> 명령어는 인터프리터에서 사용하는 디렉토리를 변경한다. 터미널에서 사용하는 것과 동일하게 사용하자. \\
:load <*.hs> 하스켈 명령어 파일을 컴파일하고 모듈을 로드한다. 이 명령어를 실행하면, 이전에 로드되었던 모듈(ex.Prelude) 대신 현재 하스켈 명령어의 모듈이 로드된다. \\
:reload 모듈을 다시 로드한다. :edit 명령어를 사용하여 편집한뒤엔 항상 리로드를 다시 해주어야 한다. \\
:set +t 항상 타입을 출력한다. \\
:quit 종료한다.



\section{Haskell 기본문법}

\subsection{주석다는법}
\begin{minted}[breaklines]{haskell}
    {-블럭주석다는법-}
    {-
        {-중첩주석-}
        여기도 주석
    -}
    
    --라인주석
\end{minted}
하스켈의 주석은 블럭주석과 라인주석으로 나뉜다. 블럭주석은 {-과-}사이에 있는 여러줄의 문자가 주석으로 처리된다. 위의 예시처럼 중첩하여 주석을 달수도 있다. 라인주석으로는 -- 바 두개를 연속으로 입력하는것으로 달수있다.  

\subsection{module}
하스켈은 모듈로서 정의되어 사용된다.
\begin{minted}[breaklines]{haskell}
    module Hello where
\end{minted}
모든 하스켈 파일의 최상단에는 모듈 이름이 정해진다. 모듈이름의 시작은 대문자로 시작되어야 한다. 위에서는 Hello로 지정하였다.

\subsection{변수}
\begin{minted}[breaklines]{haskell}
    --변수
    --변수와 함수는 무조껀 소문자로 시작해야함
    --아래는 절차형 언어와 다르게 대입이 아닌 정의
    --타입은 대부분 디폴트값을 추론해서 넣어줌
    --integer
    x = 3
    --위문장 다음에 x = 5로 재정의 불가능하다.
    
    --정의의 순서는 중요하지 않음
    --아래처럼 어딘가에 있기만 하면 괜찮다
    --패턴매칭외에는 순서 마음대로 해도 괜찮다.
    y = z + 1
    z = 8
\end{minted}
변수는 소문자로 시작해야 하며, 영어와 숫자 그리고 '(single quote) 문자를 사용할수 있다. 파이썬과 같이 따로 선언을 거치지 않고도 사용할수 있다. 다만 함수형언어인 하스켈은, 절차형 언어와는 다르게 변수를 재정의 할 수 없다. 절차형 언어에서 변수에 값을 대입하는것과는 반대로, 함수형언어에서는 변수x 를 5와 같다고 정의 하는 것이다. 두번째 예시처럼 변수의 순서는 중요하지 않다. z가 나중에 정의되어있어도 컴파일 에러가 나지 않는다.

\begin{minted}[breaklines]{haskell}

    --명시적으로 타입 명시하기
    x' :: Double
    x' = 3
    
    --출력결과
        {-
        *Hello> x'
        3.0
        *Hello> :t x'
        x' :: Double
        -}
    --출력끝
    
    --type safe한 언어이기 때문에 아래같은 정의는 불가능
    --3을 문자로 타입 추론 하는것은 불가능하다.
    x'' :: Char
    x'' = 3
\end{minted}
    명시적으로 타입을 지정할수도 있다. x = 5는 int지만 x' :: Double 로 지정한뒤 값을 정의하면 GHCi 에서 타입을 확인할때 Double로 출력된다. 단, c와 다르게 type safe한 언어이기 때문에 x''을 character등의 숫자로 나타낼수 없는 타입을  선언하고 x''에 숫자를 정의하면 컴파일 에러가 난다.

\subsection{}
\begin{minted}[breaklines]{haskell}
    module Hello where

    {-블럭주석다는법-}
    {-
    {-중첩주석-}
    여기도 주석
    -}
    --이 아래로는 모듈의 정의들
    
    --변수
    --변수와 함수는 무조껀 소문자로 시작해야함
    --나중에 배울 데이터 타입은 대문자로 시작해야함
    
    --아래는 절차형 언어와 다르게 대입이 아닌 정의
    --대부분 디폴트값을 계산해서 넣어줌
    --integer
    x = 3
    --위문장 다음에 x = 5로 재정의 불가능하다.
    
    --정의의 순서는 중요하지 않음
    --아래처럼 어딘가에 있기만 하면 괜찮다
    --패턴매칭외에는 순서 마음대로 해도 괜찮다.
    y = z + 1
    z = 8
    
    --명시적으로 타입 명시하기
    x' :: Double
    x' = 3
    {-
    *Hello> x'
    3.0
    *Hello> :t x'
    x' :: Double
    -}
    
    --type safe한 언어이기 때문에 아래같은 언어는 불가능
    --3을 문자로 타입 추론 하는것은 절대 불가능
    --x'' :: Char
    --x'' = 3
    --여기까지는 이미 화면에서 어떻게 찍을지 아는얘들
    --타입 클래스중 show 클래스 
    
    --함수정의
    --아래 f와 f'은 같다.
    f = \x -> x + x
    --위에 보단 밑에가 defualt타입으로 안변하는 안전한 방식
    f' x = x + x
\end{minted}





\bibliographystyle{plain}
\bibliography{references}
https://downloads.haskell.org/~ghc/7.0.3/docs/html/users\_guide/ghci-set.html

\end{document}

